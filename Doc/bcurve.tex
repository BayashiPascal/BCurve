\section*{Introduction}

BCurve is C library to manipulate Bezier curves of any dimension and order.\\ 

It offers function to create, clone, load, save and modify a curve, to print it, to scale, rotate (in 2D) or translate it, to get its approximate length (sum of distance between control points), to create a BCurve connecting points of a point cloud, to get the weights (coefficients of each control point given the value of the parameter of the curve), and to get the bounding box.\\ 

The library also includes a SCurve structure which is a set of BCurve (called segments) continuously connected and has the same interface as for a BCurve, plus function to add and remove segments.\\

It uses the \begin{ttfamily}PBErr\end{ttfamily}, \begin{ttfamily}PBMath\end{ttfamily}, \begin{ttfamily}GSet\end{ttfamily}, \begin{ttfamily}Shapoid\end{ttfamily} libraries.\\

\section{Definitions}

\subsection{BCurve definition}

A BCurve $B$ is defined by its dimension $D\in\mathbb{N}^*_+$, its order $O\in\mathbb{N_+}$ and its $(O+1)$ control points $\overrightarrow{C_i}\in\mathbb{R}^D$. The curve in dimension $D$ associated to the BCurve $B$ is defined by $\overrightarrow{B(t)}$:\\
\begin{equation}
\left\lbrace
\begin{array}{ll}
\overrightarrow{B(t)}=\sum_{i=0}^OW^O_i(t)\overrightarrow{C_i}&\textrm{if }t\in[0.0,1.0]\\
\overrightarrow{B(t)}=\overrightarrow{C_0}&\textrm{if }t<0.0\\
\overrightarrow{B(t)}=\overrightarrow{C_{O}}&\textrm{if }t>1.0\\
\end{array}
\right.
\end{equation}
where, if $O=0$\\
\begin{equation}
W^0_0(t)=1.0
\end{equation}
and if $O\neq 0$\\
\begin{equation}
\left\lbrace
\begin{array}{l}
W^1_0(t)=1.0-t\\
W^1_1(t)=t\\
W^i_{-1}(t)=0.0\\
W^i_j(t)=(1.0-t)W^{i-1}_j(t)+tW^{i-1}_{j-1}(t)\textrm{ for }i\in[2,O],j\in[0,i]
\end{array}
\right.
\end{equation}

\subsection{BCurve from cloud points}

Given the cloud points made of $N$ points $\overrightarrow{P_i}$, the BCurve of order $N-1$ passing through the $N$ points (in the same order $\overrightarrow{P_0}$,$\overrightarrow{P_1}$,$\overrightarrow{P_2}$,... as given in input) can be obtained as follow.\\

If $N=1$ the solution is trivial: $\overrightarrow{C_0}=\overrightarrow{P_0}$. As well, if $N=2$ the solution is trivial: $\overrightarrow{C_0}=\overrightarrow{P_0}$ and $\overrightarrow{C_1}=\overrightarrow{P_1}$.\\

If $N>2$, we need first to define the $N$ values $t_i$ corresponding to each $\overrightarrow{P_i}$ ($\overrightarrow{B(t_i)}=\overrightarrow{P_i}$). We will consider here $t_i$ such as\\
\begin{equation}
t_i=\frac{L(\overrightarrow{P_i})}{L(\overrightarrow{P_{N-1}})}
\end{equation}
where
\begin{equation}
\left\lbrace
\begin{array}{l}
L(P_0)=0.0\\
L(P_i)=\sum^i_{j=1}\left|\left|\overrightarrow{P_{j-1}P_j}\right|\right|\\
\end{array}
\right.
\end{equation}
then we can calculate the $C_i$ as follow. We have $\overrightarrow{C_0}=\overrightarrow{P_0}$ and $\overrightarrow{C_{N-1}}=\overrightarrow{P_{N-1}}$, and others $\overrightarrow{C_i}$ can be obtained by solving the linear system below for each dimension:\\
\begin{equation}
\begin{array}{c}
\left[
\begin{array}{ccc}
W^{N-1}_1(t_1)&...&W^{N-1}_{N-2}(t_1)\\
...&...&...\\
W^{N-1}_1(t_{N-2})&...&W^{N-1}_{N-2}(t_{N-2})\\
\end{array}
\right]\left[
\begin{array}{c}
C_1\\
...\\
C_{N-2}\\
\end{array}
\right]=\\
\\
\left[
\begin{array}{c}
P_1-\left(W^{N-1}_0(t_1)P_0+W^{N-1}_{N-1}(t_1)P_{N-1}\right)\\
...\\
P_{N-2}-\left(W^{N-1}_0(t_{N-2})P_0+W^{N-1}_{N-1}(t_{N-2})P_{N-1}\right)\\
\end{array}
\right]
\end{array}
\end{equation}

\section{Interface}

\begin{scriptsize}
\begin{ttfamily}
\verbatiminput{/home/bayashi/Coding/BCurve/bcurve.h}
\end{ttfamily}
\end{scriptsize}

\section{Code}

\subsection{bcurve.c}

\begin{scriptsize}
\begin{ttfamily}
\verbatiminput{/home/bayashi/Coding/BCurve/bcurve.c}
\end{ttfamily}
\end{scriptsize}

\subsection{bcurve-inline.c}

\begin{scriptsize}
\begin{ttfamily}
\verbatiminput{/home/bayashi/Coding/BCurve/bcurve-inline.c}
\end{ttfamily}
\end{scriptsize}

\section{Makefile}

\begin{scriptsize}
\begin{ttfamily}
\verbatiminput{/home/bayashi/Coding/PBMath/Makefile}
\end{ttfamily}
\end{scriptsize}

\section{Unit tests}

\begin{scriptsize}
\begin{ttfamily}
\verbatiminput{/home/bayashi/Coding/BCurve/main.c}
\end{ttfamily}
\end{scriptsize}

\section{Unit tests output}

\begin{scriptsize}
\begin{ttfamily}
\verbatiminput{/home/bayashi/Coding/BCurve/unitTestRef.txt}
\end{ttfamily}
\end{scriptsize}

bcurve.txt:\\

\begin{scriptsize}
\begin{ttfamily}
\verbatiminput{/home/bayashi/Coding/BCurve/bcurve.txt}
\end{ttfamily}
\end{scriptsize}

scurve.txt:\\

\begin{scriptsize}
\begin{ttfamily}
\verbatiminput{/home/bayashi/Coding/BCurve/scurve.txt}
\end{ttfamily}
\end{scriptsize}


