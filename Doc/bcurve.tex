\section*{Introduction}

BCurve is C library to manipulate geometry based on Bezier curves of any dimension and order.\\ 

It offers function to create, clone, load and save (JSON format), and modify a curve or surface, to print it, to scale, rotate (in 2D) or translate it, to get the weights (coefficients of each control point given the value of the parameter of the curve), and to get the bounding box.\\ 

BCurve objects are Bezier curves from 1D to ND. For BCurve object, the library offers functions to get its approximate length (sum of distance between control points), and to create a BCurve connecting points of a point cloud.\\

SCurve objects are a set of BCurve (called segments) continuously connected and has the same interface as a BCurve, plus function to add and remove segments.\\

BBody objects are extension of BCurve objects for the case M dimensions to N dimensions. If M equals 1 it is equivalent to a BCurve. If M equals 2 it is equivalent to a surface in N dimension. If M equals 3 it is equivalent ot a volume. Note that by using one dimension as the time dimension one can describes the movement of a curve, surface, etc... over time. The library offers the same functions for a BBody as for a BCurve.\\ 

It uses the \begin{ttfamily}PBErr\end{ttfamily}, \begin{ttfamily}PBMath\end{ttfamily}, \begin{ttfamily}GSet\end{ttfamily}, \begin{ttfamily}Shapoid\end{ttfamily} libraries.\\

\section{Definitions}

\subsection{BCurve definition}

A BCurve $B$ is defined by its dimension $D\in\mathbb{N}^*_+$, its order $O\in\mathbb{N_+}$ and its $(O+1)$ control points $\overrightarrow{C_i}\in\mathbb{R}^D$. The curve in dimension $D$ associated to the BCurve $B$ is defined by $\overrightarrow{B(t)}$:\\
\begin{equation}
\left\lbrace
\begin{array}{ll}
\overrightarrow{B(t)}=\sum_{i=0}^OW^O_i(t)\overrightarrow{C_i}&\textrm{if }t\in[0.0,1.0]\\
\overrightarrow{B(t)}=\overrightarrow{C_0}&\textrm{if }t<0.0\\
\overrightarrow{B(t)}=\overrightarrow{C_{O}}&\textrm{if }t>1.0\\
\end{array}
\right.
\end{equation}
where, if $O=0$\\
\begin{equation}
W^0_0(t)=1.0
\end{equation}
and if $O\neq 0$\\
\begin{equation}
\left\lbrace
\begin{array}{l}
W^1_0(t)=1.0-t\\
W^1_1(t)=t\\
W^i_{-1}(t)=0.0\\
W^i_j(t)=(1.0-t)W^{i-1}_j(t)+tW^{i-1}_{j-1}(t)\textrm{ for }i\in[2,O],j\in[0,i]
\end{array}
\right.
\end{equation}

\subsection{BCurve from cloud points}

Given the cloud points made of $N$ points $\overrightarrow{P_i}$, the BCurve of order $N-1$ passing through the $N$ points (in the same order $\overrightarrow{P_0}$,$\overrightarrow{P_1}$,$\overrightarrow{P_2}$,... as given in input) can be obtained as follow.\\

If $N=1$ the solution is trivial: $\overrightarrow{C_0}=\overrightarrow{P_0}$. As well, if $N=2$ the solution is trivial: $\overrightarrow{C_0}=\overrightarrow{P_0}$ and $\overrightarrow{C_1}=\overrightarrow{P_1}$.\\

If $N>2$, we need first to define the $N$ values $t_i$ corresponding to each $\overrightarrow{P_i}$ ($\overrightarrow{B(t_i)}=\overrightarrow{P_i}$). We will consider here $t_i$ such as\\
\begin{equation}
t_i=\frac{L(\overrightarrow{P_i})}{L(\overrightarrow{P_{N-1}})}
\end{equation}
where
\begin{equation}
\left\lbrace
\begin{array}{l}
L(P_0)=0.0\\
L(P_i)=\sum^i_{j=1}\left|\left|\overrightarrow{P_{j-1}P_j}\right|\right|\\
\end{array}
\right.
\end{equation}
then we can calculate the $C_i$ as follow. We have $\overrightarrow{C_0}=\overrightarrow{P_0}$ and $\overrightarrow{C_{N-1}}=\overrightarrow{P_{N-1}}$, and others $\overrightarrow{C_i}$ can be obtained by solving the linear system below for each dimension:\\
\begin{equation}
\begin{array}{c}
\left[
\begin{array}{ccc}
W^{N-1}_1(t_1)&...&W^{N-1}_{N-2}(t_1)\\
...&...&...\\
W^{N-1}_1(t_{N-2})&...&W^{N-1}_{N-2}(t_{N-2})\\
\end{array}
\right]\left[
\begin{array}{c}
C_1\\
...\\
C_{N-2}\\
\end{array}
\right]=\\
\\
\left[
\begin{array}{c}
P_1-\left(W^{N-1}_0(t_1)P_0+W^{N-1}_{N-1}(t_1)P_{N-1}\right)\\
...\\
P_{N-2}-\left(W^{N-1}_0(t_{N-2})P_0+W^{N-1}_{N-1}(t_{N-2})P_{N-1}\right)\\
\end{array}
\right]
\end{array}
\end{equation}

\subsection{BBody definition}

A BBody $A$ is defined by its input dimension $D_i\in\mathbb{N}^*_+$, its output dimension $D_o\in\mathbb{N}^*_+$, its order $O\in\mathbb{N_+}$ and its $(O+1)^{D_i}$ control points $\overrightarrow{C_i}\in\mathbb{R}^{D_o}$. Control points indices are ordered as follow (for an example BBody with $D_i=3$): (0,0,0),(0,0,1),...,(0,0,O+1),(0,1,0),(0,1,1),... \\
Note that if $D_i$ is equal to 1, a BBody is equivalent to a BCurve.\\
The function $\overrightarrow{A}():[0.0,1.0]^{D_i}\mapsto\mathbb{R}^{D_o}$ associated to the BBody $A$ is defined by:\\
\begin{equation}
\overrightarrow{A}(\overrightarrow{u})=\overrightarrow{R_A}(\overrightarrow{0},\overrightarrow{u},0)
\end{equation}
where
\begin{equation}
\left\lbrace
\begin{array}{ll}
\overrightarrow{R_A}(\overrightarrow{c},\overrightarrow{u},d)=\overrightarrow{B_{\lbrace\overrightarrow{C}_{I(\overrightarrow{c},d)}\rbrace}}(u_d)&\textrm{if }d=D_i-1\\
\overrightarrow{R_A}(\overrightarrow{c},\overrightarrow{u},d)=\overrightarrow{B_{\lbrace\overrightarrow{R_S}(\lbrace\overrightarrow{c}\rbrace_d,\overrightarrow{u},d+1)\rbrace}}(u_d)&\textrm{if }d\ne D_i-1\\
\end{array}
\right.
\end{equation}
where $\overrightarrow{B_{\lbrace\bullet\rbrace}}$ is the BCurve of dimension $D_o$, order $O$ and control points $\bullet$. And $\lbrace\overrightarrow{C}_{I(\overrightarrow{c},d)}\rbrace$ is the set of control points of S of indices:\\
\begin{equation}
\lbrace I(\overrightarrow{c},d)\rbrace=\lbrace
\begin{array}{l}
\sum_{i\in[0,D_i-1]|i\ne d}\left(O^{(D_i-1-i)}c_i\right)+O^{(D_i-1-d)}j
\end{array}
\rbrace_{j\in[0,O]}
\end{equation}
and $\lbrace\overrightarrow{R_A}(\lbrace\overrightarrow{c}\rbrace_d,\overrightarrow{u},d')\rbrace$ is the set of intermediate control points calculated for:\\
\begin{equation}
\lbrace\overrightarrow{c}\rbrace_d=\lbrace(\overrightarrow{c_0,c_1,..,c_{d-1},j,c_{d+1},..,c_{D_i-1}})\rbrace_{j\in[0,O]}
\end{equation}

\section{Interface}

\begin{scriptsize}
\begin{ttfamily}
\verbatiminput{/home/bayashi/Coding/BCurve/bcurve.h}
\end{ttfamily}
\end{scriptsize}

\section{Code}

\subsection{bcurve.c}

\begin{scriptsize}
\begin{ttfamily}
\verbatiminput{/home/bayashi/Coding/BCurve/bcurve.c}
\end{ttfamily}
\end{scriptsize}

\subsection{bcurve-inline.c}

\begin{scriptsize}
\begin{ttfamily}
\verbatiminput{/home/bayashi/Coding/BCurve/bcurve-inline.c}
\end{ttfamily}
\end{scriptsize}

\section{Makefile}

\begin{scriptsize}
\begin{ttfamily}
\verbatiminput{/home/bayashi/Coding/PBMath/Makefile}
\end{ttfamily}
\end{scriptsize}

\section{Unit tests}

\begin{scriptsize}
\begin{ttfamily}
\verbatiminput{/home/bayashi/Coding/BCurve/main.c}
\end{ttfamily}
\end{scriptsize}

\section{Unit tests output}

\begin{scriptsize}
\begin{ttfamily}
\verbatiminput{/home/bayashi/Coding/BCurve/unitTestRef.txt}
\end{ttfamily}
\end{scriptsize}

bcurve.txt:\\

\begin{scriptsize}
\begin{ttfamily}
\verbatiminput{/home/bayashi/Coding/BCurve/bcurve.txt}
\end{ttfamily}
\end{scriptsize}

scurve.txt:\\

\begin{scriptsize}
\begin{ttfamily}
\verbatiminput{/home/bayashi/Coding/BCurve/scurve.txt}
\end{ttfamily}
\end{scriptsize}

bbody.txt:\\

\begin{scriptsize}
\begin{ttfamily}
\verbatiminput{/home/bayashi/Coding/BCurve/bbody.txt}
\end{ttfamily}
\end{scriptsize}

